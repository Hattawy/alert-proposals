\addcontentsline{toc}{chapter}{Abstract}
{\large\textbf{Run group abstract:}}
\newline

We propose a comprehensive physics program to investigate the quark and
gluon structure of light nuclei, namely deuterium and $^4$He, through coherent and
incoherent exclusive and semi-inclusive Deep Virtual Compton Scattering (DVCS) and
Deep Virtual Meson Production (DVMP). Coherent DVCS off nuclei is a particularly
powerful tool for nuclear tomography through the access of partons position in the
transverse plane. An additional focus of this program is the study of the effect of the
nuclear medium on the structure of nucleons. We propose a next generation nuclear
physics measurements in which low energy recoil nuclei are detected. The tagging of
recoiling nuclei especially in semi-inclusive reactions will be realized for the first time in
these measurements. This powerful technique will provide unique information about the
nature of medium modification including the EMC effect through its dependence on the
nucleon virtuality. Finally, we propose to measure tagged DVCS on light nuclei both for
quasi-free neutron and bound nucleon GPDs. In both cases, we want to study nuclear
effects and their manifestation in GPDs including the effect of Final State Interactions in
the measurements of the bound nucleon beam spin asymmetries.

At the heart of this program is the Low Energy Recoil Tracker (ALERT) in addition
to the CLAS12 detector. The ALERT detector is composed of a stereo drift chamber for
track reconstruction and an array of scintillators for particle identification. Coupling these
two types of fast detectors ensure that ALERT detector can be included in the trigger to
reject background efficiently, while keeping the material budget as low as possible to
detect low energy particles. ALERT will be installed inside the solenoid magnet instead
of the CLAS12 Silicon Vertex Tracker. We will use a gas target straw filled with
deuterium or $^4$He at 3 atm.

The ALERT run group needs 11 GeV longitudinally polarized electron beam.
Although the two main targets are $^4$He and deuterium, we also need to run hydrogen and
$^4$He targets at different beam energies for detector calibration.

\newpage
{\large\textbf{Proposal abstract:}}
\newline

We propose to study the partonic structure of $^4$He and deuterium by measuring
the Beam Spin Asymmetry (BSA) in both coherent DVCS and DVMP. In the latter,
coherent production of neutral pions and phi mesons will be measured. Despite its
simple structure, a light nucleus such as $^4$He has a density and a binding energy
comparable to that of heavier nuclei. Therefore, studying $^4$He nucleus, one can learn
typical features of the partonic structure of atomic nuclei. In addition, due to its spin-0,
only one chiral-even GPD, $H (x, \xi , t)$ , and one chiral-odd, $H_T (x, \xi , t)$ , parameterize its
partonic structure at twist-2 while two GPDs, $H_A^{twist-3} (x, \xi , t)$ and
arise at twist-3. The latter describes partonic spin-orbit correlations in the nucleus.


A major goal of this proposal is to cover a wide kinematical range and collect
higher statistics leveraging the knowledge obtained during eg6 running, where, for the
first time, exclusive coherent DVCS off $^4$He was successfully measured in the CLAS E-
08-024 experiment. The real and imaginary parts of the $^4$He Compton Form factors
(CFFs) will be extracted in a model independent way from the experimental
asymmetries, allowing us to access the nuclear transverse spatial distributions of
partons and their spin correlations. The spin-1 nature of the deuterium, however, leads
to nine GPDs, at leading twist. However, only a combination of the three GPDs ($H_1$, $H_3$
and $H_5$) is accessible through BSA measurements. Through these measurements, we
will have a first direct sensitivity to the GPD $H_3$, which is related to the quadrupole form
factor of the deuteron.


The other focus of this proposal is to measure exclusive coherent phi meson
electroproduction off a $^4$He target. The kinematic regime to be explored includes very low
$|t|$ up to the first diffractive minimum as found in $^4$He elastic 
scattering ($|t^\prime| \simeq 0.6 GeV^2$), $Q^2$
up to 12 GeV$^2$, and $x_B$ up to 0.4. The phi meson will be detected primarily through the
charged $K^+ K^-$ channel, with the neutral $K^0_S K^0_L$ channel also available through
$K_S \rightarrow \pi^+ \pi^-$. Differential cross-sections for phi electroproduction off $^4$He will be
measured for the first time.

The ALERT detector provides a unique opportunity to study the gluonic structure
of a dense light nucleus. The average transverse gluonic density of the $^4$He nucleus can
be extracted within a GPD framework using the measured longitudinal cross-section of
coherent phi production. Additionally, threshold effects of phi production can be explored
by exploiting the ALERT detector's large transverse acceptance of low $|t|$ events. This
experiment will complement the previously approved experiment E12-12-007 that will
study the gluonic distribution of the proton using a very similar framework.

\newpage

