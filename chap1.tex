
\chapter{Physics Motivations}
\label{chap:physics}

\section{DVCS and DVMP Reactions}
~~~~More information on the hadronic structure lies in the correlation between 
the momentum and the spatial degrees of freedom of the constituent partons.  
Such correlations are accessible via the GPDs. The GPDs contain information on 
the correlation between longitudinal momenta and transverse positions of the 
partons. Also they contain information on the quark-antiquark structure of the 
target. The GPDs are accessible via hard exclusive reactions, like Deeply 
Virtual Compton Scattering (DVCS) and Deeply Virtual Meson Production (DVMP).  
The DVCS process is the exclusive electroproduction of a real photon from a 
quark of the nucleon.

The twist is defined as the dimension of an operator minus its spin. The 
leading twist of DVCS is n~=~2, and higher twists are suppressed by a power of 
$(M_{N}/Q)^{n-2}$ \cite{twist_def}. The DVCS reaction ($eH\rightarrow 
e'H'\gamma$) at leading twist (twist-2) and at leading order in the strong 
coupling constant of QCD ($\alpha _{s}$), is described by the handbag diagram 
shown in figure \ref{fig:DVCS}. In the DVCS process, a highly virtual photon 
($\gamma^{*}$), radiated by the incident electron ($e$), interacts with a quark 
which emits a real photon ($\gamma$) before going back to the nucleon. The 
final-state photon of the DVCS process is replaced by a final state meson in 
the case of the DVMP reaction.

\begin{figure}[t]
\begin{minipage}[c]{.46\linewidth}
\vspace{-0.2in}
\centering
\includegraphics[height=5.5cm]{fig_NuclGPD/DVCS2.png}
\caption{Schematic of the leading-twist handbag diagram of DVCS on a hadron, at 
the leading order in the coupling constant ($\alpha _{s}$).} \label{fig:DVCS}
\end{minipage} \hfill
\begin{minipage}[c]{.46\linewidth}
\includegraphics[height=5.5cm]{fig_NuclGPD/DVMP.png}
\caption{A schematic for the Handbag diagram of DVMP on a hadron at leading 
twist order.}
\label{fig:DVMP}
\end{minipage}
\end{figure}
    
  The hard reactions are characterized by the factorization property. With this 
property, the handbag diagrams in figure \ref{fig:DVCS} can be factorized into 
two parts, a hard and a soft part. The hard part is calculable by perturbative 
methods, while the soft part is non-perturbative and is parametrized in terms 
of the GPDs. This factorization has been proven for the DVCS amplitude by two 
independent calculations: one by John Collins and Andreas Freund 
\cite{Freund_Collins}, and one by Xiangdong Ji and Jonathan Osborne 
\cite{Ji_Osborne}.

The GPDs depend on three variables: $x$, $\xi$ and $t$.  $x+\xi$ is the 
nucleon's longitudinal momentum fraction carried by the struck quark, 2$\xi$ is 
the longitudinal momentum fraction of the momentum transfer $\Delta$ ($= p' - 
p$), and $t$ (=$\Delta^{2}$) is the squared momentum transfer between the 
initial and the final states of the nucleon. One can define a 
GPD~($x$,$\xi$,$t$) as the probability amplitude of picking up a parton with a 
longitudinal momentum $x+\xi$ and putting it back in the nucleon with a 
longitudinal momentum $x-\xi$, without breaking the nucleon, at a squared 
momentum transfer $t$. 

In the infinite-momentum frame, where the initial and the final nucleons go at 
the speed of light along the positive z-axis, the partons have relatively small 
transverse momenta compared to their longitudinal momenta. Referring to figure 
\ref{fig:DVCS}, the struck parton carries a longitudinal momentum fraction 
$x+\xi$ and it goes back into the nucleon with a momentum fraction $x-\xi$. The 
GPDs are defined in the interval where $x$ and $\xi$~$\in$ [-1,1], which can be 
separated into three regions as can be seen in figure \ref{DGLAP}.  The regions 
are:
\begin{itemize}
\item $x\in$ [$\xi$,1]: both momentum fractions $x+\xi$ and $x-\xi$ are 
   positive and the process describes the emission and reabsorption of a quark.
\item $x\in $ [-$\xi$,$\xi$]: $x+\xi$ is positive reflecting the emission of a 
   quark, while $x-\xi$ is negative and is interpreted as an antiquark being 
   emitted from the initial proton.
\item $x\in$ [-1,-$\xi$]: both fractions are negative, and $x+\xi$ and $x-\xi$ 
   represent the emission and reabsorption of antiquarks.
   \end{itemize}

The GPDs in the first and in the third regions represent the probability 
amplitude of finding a quark or an antiquark in the nucleon, while in the 
second region they represent the probability amplitude of finding a 
quark-antiquark pair in the nucleon \cite{Diehl:2001pm}.\\
  
Following the definition of reference \cite{Ji:1998pc}, the differential DVCS 
cross section is obtained from the DVCS scattering amplitude 
($\mathcal{T}_{DVCS}$) as:
\begin{equation}
 \frac{d^{5}\sigma}{dQ^{2}\, dx_{B}\, dt\, d\phi\, d\phi_{e}} = 
 \frac{1}{(2\pi^{4})32}\frac{x_{B}\, 
 y^{2}}{Q^{4}}\bigg(1+\frac{4M^{2}x_{B}^{2}}{Q^{2}}\bigg)^{-1/2} 
 |\mathcal{T}_{DVCS}|^{2},
\label{DVCSCrossSection_tot}
\end{equation}
where $\phi_{e}$ is the azimuthal angle of the scattered lepton, 
$y=\frac{E-E'}{E}$ and $Q^{2},x_{B}, t, \phi$ are the four kinematic variables 
that describe the process. The variable $\phi$ is the angle between the 
leptonic and the hadronic planes, as can be seen in figure \ref{fig:phi}.\\

\begin{figure}[tb]
\centering
\includegraphics[scale=0.37]{fig_NuclGPD/DGLAP.png}
\caption{The parton interpretations of the GPDs in three $x$-intervals 
[-1,-$\xi$], [-$\xi$,$\xi$] and [$\xi$,1]. The red arrows indicate the initial 
and the final-state of the proton, while the blue (black) arrows represent 
helicity (momentum) of the struck quark.} \label{DGLAP}
\end{figure}

\begin{figure}[tb]
\centering
\includegraphics[scale=0.30]{fig_NuclGPD/plane.png}
\caption{The definition of the azimuthal angle $\phi$ between the leptonic and 
the hadronic planes.} \label{fig:phi}
\end{figure}



\subsection{Nuclear DVCS}
~~~~Nuclear targets provide access to the measurement of two DVCS channels: the 
coherent and the incoherent. In the coherent channel, the target nucleus 
remains intact and recoils as a whole while emitting a real photon ($eA 
\rightarrow e' A' \gamma$). This process allows to measure the nuclear GPDs of 
the target, which contain information on the partons correlations and the 
nuclear forces in the target \cite{Polyakov:2002yz, Liuti:2005qj}. In the 
incoherent channel, the nucleus breaks up and the DVCS takes place on a bound 
nucleon that emits the final photon ($eA \rightarrow e' N' \gamma$ X). The 
latter allows to measure the GPDs of the bound nucleons and study the medium 
modifications of the nucleons in the nuclear medium. Figure 
\ref{fig:nuclear_DVCS} shows the diagrams of the two DVCS channels.
\begin{figure}[tbp]
\centering
\includegraphics[scale=0.3]{fig_NuclGPD/nuclear_DVCS.png}
\caption{ The leading twist handbag diagrams of the two DVCS channels from a 
nuclear target, coherent channel (on the left) and incoherent channel (on the 
right). } \label{fig:nuclear_DVCS}
\end{figure}

The GPDs depend on three variables: $x$, $\xi$ and $t$. $x+\xi$ is the 
nucleon's longitudinal momentum fraction carried by the struck quark, 2$\xi$ is 
the longitudinal momentum fraction of the momentum transfer $\Delta$ ($= p' - 
p$), and $t$ (=$\Delta^{2}$) is the squared momentum transfer between the 
initial and the final states of the hadron target. Experimentally, only $\xi$ 
and $t$ are measurable in the DVCS reaction. At twist-2 order, $\xi$ can be 
calculated as $x_B/(2-x_B)$, where $x_B$ is the Bjorken variable $(= 
Q^{2}/(2M_{N}(E-E')$ with $Q^2$ is the vertuality of the exchanged photon, 
$M_{N}$ is the mass of the nucleon and $E(E')$ is the energy of the incident 
(scattered) electron.

The number of GPDs needed to parametrize the partonic structure of a nucleus 
depends on the different configurations between the spin of the nucleus and the 
helicity direction of the struck quark.  In principle, for a target of spin 
$s$, the number of the chiral-even GPDs is equal to ($2s+1$)$^2$ for each quark 
flavor. The DVCS off spinless nuclear targets, such as $^4He$, $^{12}C$ and 
$^{16}O$, is simpler to study as only one GPD ($H_{A}(x,\xi,t)$) arises at 
leading twist to parametrize their partonic structure.

Nuclear DVCS provides a quantitative information on the nuclear medium effects, 
the quark confinement size of the bound nucleons, see figure 
\ref{fig:quarks_nucleus}. The Fourier transform of the nucleon GPDs over the 
momentum transfer $\Delta$ gives the transverse separation ($b'$) between 
quarks in the nucleon, while the transform of the nuclear GPD 
($H_{A}(x,\xi,t)$) gives the transverse separation ($b$) between the quarks in 
the nucleus. Knowing these two separations, one can access the transverse 
separation ($\beta = b - b'$) between the nucleons in a nucleus 
\cite{Liuti:2005qj}.

\begin{figure}[tbp]
\centering
\includegraphics[scale=0.6]{fig_NuclGPD/quarks_nucleus.png}
\vspace{-0.7in}
\caption{The spatial coordinates of quarks in a nucleus. See main text for 
definition of the variables.}
\label{fig:quarks_nucleus}
\end{figure}

The $^4$He nucleus shows a clear EMC effect. This nucleus is characterized by 
its spin-zero, a high density and it is a well-known few-body system. These 
aspects make the $^4$He nucleus an ideal target to be considered for the 
understanding of the nuclear effects at the partonic level. 
  
The $^4$He GPD $H_{A}(x,\xi,t)$ is characterized by:
\begin{itemize}
 \item The universality of $H_{A}$: the $H_{A}$ describes the partonic 
 structure of the $^4$He in a DVCS reaction the same way as in a DVMP reaction.  
 \item In the forward limit ($t\rightarrow 0$), $H_{A}$ is reduced to the usual 
    PDF of $^4$He that is accessible via DIS.
 \item $H_{A}$ can be decomposed into a polynomial in $\xi$.
 \item The first moment of $H_{A}$ is the $^4$He elastic electromagnetic form 
    factor $F_{A}(t)$, such as:
 \begin{equation}
   \sum_{q} \int_{-1}^{1} dx ~H_{A}^{q}(x, \xi, t) =  F_{A}(t),
\end{equation}
  where the sum runs over all the quark flavors.
  \item The second moment of $H_{A}^{q}(x, \xi, t)$ reads
  \begin{equation}
 \int_{-1}^{1} dx ~x H_{A}^{q}(x, \xi, t) = M ^{q/A}_{2}(t) + \frac{4}{5} 
 \xi^{2} d^{q/A}_{2}(t)
  \end{equation}
   where the first term of the right-hand side represents the momentum fraction 
carried by each quark flavor q, and the second term is encoding information 
about the forces experienced by partons inside the nuclei 
\cite{Polyakov:2002yz}.  \item $H_{A}$ is not directly measured from 
experiment, but we measure its corresponding Compton form factor 
$\mathcal{H}_{A}$.  \end{itemize}  

%~~~~The $^4$He DVCS amplitude can be expressed as \cite{Kirchner:2003wt}:
% \begin{equation}
%\mathcal{T}_{DVCS} \propto \sum_{q} e^{2}_{q} \mathcal{P} \int_{-1}^{1} dx 
%\left(\frac{1}{x-\xi} + \frac{1}{x+\xi} \right) H^{q}_{A}(x,\xi,t) \\- i\pi 
%\sum_{q} \left( e^{2}_{q} \left[ H^{q}_{A}(\xi,\xi,t) - H^{q}_{A}(-\xi,\xi,t) 
%\right] \right),
%\end{equation} where the first term on the right-hand side stands for the real 
%part of the CFF $\mathcal{H}_{A}$, while the second term for the imaginary 
%part of $\mathcal{H}_{A}$. 


\begin{figure}[tbp]
\centering
\includegraphics[scale=0.26]{fig_NuclGPD/BH.png}
\caption{Schematic for the Bethe-Heitler process. The final real photon can be  
emitted from the incoming electron (left plot) or from the scattered electron 
(right plot).} 
\label{BH}
\end{figure}

\begin{figure}[tp]
\begin{minipage}[c]{.46\linewidth}
\hspace{-0.2in}\includegraphics[height=6.0cm]{fig_NuclGPD/He-4_FF.png}
\caption{$^4$He charge form factor measurements at Stanford, SLAC, Orsay, Mainz 
and JLab Hall A compared with theoretical calculations. The figure is from 
\cite{PhysRevLett.112.132503}. }
\label{fig:He-4_FFs}
\end{minipage} \hfill
\begin{minipage}[c]{.46\linewidth}
\hspace{-0.3in}\includegraphics[height=6.1cm]{fig_NuclGPD/BH_He-4.png}
\caption{The calculated BH cross section as a function of $\phi$ on a $^4He$ 
target at three values of $x_{B}$ and fixed values of $Q^{2}$ and $t$.  
($t$~=~-~0.1~GeV$^2$/c$^2$ corresponds to $Q^2$~$\approx$~2.57~fm$^{-2}$ on 
figure \ref{fig:He-4_FFs}).}
\vspace{+0.3in}
\label{fig:BH_cross_section_4He}
\end{minipage}
\end{figure}




Experimentally, the DVCS reaction is indistinguishable from the Bethe-Heitler 
(BH) process, which is the reaction where the final photon is emitted either 
from the incoming or the outgoing leptons, as shown in figure \ref{BH}. The BH 
process is not sensitive to GPDs and does not carry information about the 
partonic structure of the hadronic target. The BH cross section is calculable 
from the well-known electromagnetic FFs. The DVCS amplitude is enhanced through 
the interference with the BH process, that is calculable from the well-known 
elastic FF. Figure \ref{fig:He-4_FFs} shows the world measurements of the 
$^4$He $F_{A}(t)$ along with theoretical calculations. Following the $F_{A}(t)$ 
parametrization by R.  Frosch and his collaborators \cite{PhysRev.160.874} 
(valid at the small values of $-t$ which are of interest in this work), figure 
\ref{fig:BH_cross_section_4He} shows the calculated BH as a function of the 
azimuthal angle between the leptonic and the hadronic planes ($\phi$), using a 
6~GeV electron beam on a $^4$He target.\\

At leading twist, the differential cross section for a longitudinally-polarized 
electron beam ($\lambda$) and an unpolarized $^4$He target can be decomposed 
into BH, DVCS, and interference terms as:
\small
\begin{equation}
\frac{d^{5}\sigma_{\lambda}}{dx_{A} dQ^{2} dt d\phi_{e} d\phi} = 
\frac{\alpha^{3}}{16 \pi^{2}} \frac{x_{A} \, y^{2}}{Q^{4} \sqrt{1 + \epsilon ^{2}}} 
\frac{
|\mathcal{T}_{BH}|^{2} + |{\mathcal{T}}_{DVCS}^{\lambda}|^{2} + 
{\mathcal{I}}_{BH*DVCS}^{\lambda}}{e^{6}}
\label{eq:sigdiff}
\end{equation}
\normalsize
where $y = \frac{p \cdot q}{p \cdot k}$, $\epsilon  =  \frac{2 x_{A} M_{A}}{Q}$ 
and $x_A  =  \frac{Q^2}{2 p \cdot q}$, and $M_{A}$ being the mass of the 
$^4$He.

The nuclear BH, DVCS and interference scattering amplitudes can be decomposed 
into a finite series of Fourier harmonics as expressed in  equations 
\ref{TTBH}, \ref{TTDVCS} and \ref{TTinter}.\\
The different amplitudes are written as \cite{Belitsky:2008bz}:
\small
\begin{equation}
 |\mathcal{T}_{BH}|^{2} =  \frac{e^{6} (1 + \epsilon^{2})^{-2}}{x^{2}_{A} y^{2} 
 t \mathcal{P}_{1}(\phi) \mathcal{P}_{2}(\phi)} \left[ c_{0}^{BH} + c_{1}^{BH} 
 \cos(\phi) + c_{2}^{BH} \cos(2\phi)\right] \label{TTBH}
\end{equation}

\begin{equation}
 |\mathcal{T}_{DVCS}|^{2} =  \frac{e^{6}}{y^{2} Q^{2}} \left[ c_{0}^{DVCS} + 
 \sum_{n=1}^{2} \Bigg( c_{n}^{DVCS} \cos(n \phi) + \lambda s_{n}^{DVCS} \sin(n 
 \phi)\Bigg) \right] \label{TTDVCS}
\end{equation}

\begin{equation}
 \mathcal{I}_{BH*DVCS} =  \frac{\pm e^{6}}{x_A y^{3} t \, \mathcal{P}_{1}(\phi) 
 \mathcal{P}_{2}(\phi)} \left[ c_{0}^{I} + \sum_{n=0}^{3} \Bigg( c_{n}^{I} 
 \cos(n \phi) + \lambda s_{n}^{I} \sin(n \phi) \Bigg) \right] \label{TTinter} 
 \end{equation}
The explicit expressions of the coefficients can be found in Appendix 
\ref{app:Helium_cross_section}.\\


~~~~It is convenient to use the beam-spin asymmetry as DVCS observable because 
most of the experimental normalization and acceptance issues cancel out in an 
asymmetry ratio. The beam-spin asymmetry is measured using a polarized lepton 
beam on an unpolarized target (U). JLab provides a longitudinally (L) polarized 
electron beam, $P_{B}~\approx$~85~$\%$. It is defined as:
  \begin{equation}
  A_{LU} = \frac{d^{5}\sigma^{+} - d^{5}\sigma^{-} }
                {d^{5}\sigma^{+} + d^{5}\sigma^{-}}.
    \label{BSA_equation}
  \end{equation}
 where $d^{5}\sigma^{+}$($d^{5}\sigma^{-}$) is the DVCS differential cross 
 section for a positive (negative) beam helicity.\\

 At leading twist, the beam-spin asymmetry ($A_{LU}$) with the  two opposite 
 helicities of a  longitudinally-polarized electron beam (L) on a spin-zero 
 target (U) can be written as:        \begin{eqnarray}
A_{LU}& =& \frac{x_A(1+\epsilon^2)^2}{y} \, s_1^{INT} \sin(\phi) \, 
\bigg/ \, \bigg[ \, \sum_{n=0}^{n=2}c_n^{BH}\cos{(n\phi)} +  \\
& & \frac{x_A^2 t {(1+\epsilon^2)}^2}{Q^2} P_1(\phi) P_2(\phi) \, c_0^{DVCS} + 
\frac{x_A (1+\epsilon^2)^2}{y} \sum_{n=0}^{n=1} c_n^{INT} \cos{(n\phi)} \bigg].  \nonumber 
\label{eq:coh_BSA}
\end{eqnarray}

where $\mathcal{P}_1(\phi)$ and $\mathcal {P}_2(\phi)$ are the Bethe-Heitler 
propagators. The factors: $c_{0,1,2}^{BH}$, $c_0^{DVCS}$, $c_{0,1}^{INT}$ and 
$s_1^{INT}$ are the Fourier coefficients of the BH, the DVCS and the 
interference amplitudes for a spin-zero target \cite{Kirchner:2003wt}.

~~~~The beam-spin asymmetry ($A_{LU}$) can be rearranged as
\begin{equation}
A_{LU}(\phi) = \frac{\alpha_{0}(\phi) \, \Im m(\mathcal{H}_{A})}
{\alpha_{1}(\phi) + \alpha_{2}(\phi) \, \Re e(\mathcal{H}_{A}) + \alpha_{3}(\phi) \, 
\big( 
\Re e(\mathcal{H}_{A})^{2} + \Im m(\mathcal{H}_{A})^{2} \big)}
\label{eq:A_LU-coh}
\end{equation}
where $\Im m(\mathcal{H}_{A})$ and $\Re e(\mathcal{H}_{A})$ are the imaginary 
and real parts of the CFF $\mathcal{H}_{A}$ associated to the GPD $H_A$. The 
$\alpha_{i}$'s are $\phi$-dependent kinematical factors that depend on the 
nuclear form factor $F_A$ and the independent variables $Q^2$, $x_{B}$ and $t$.  
These factors are simplified as:

\small
\begin{eqnarray}
   \alpha_0 (\phi) & = &\frac{x_{A}(1+\epsilon^2)^2}{y} S{++}(1) \sin(\phi)\\
    \alpha_1 (\phi) & = & c_0^{BH}+c_1^{BH} \cos({\phi})+c_2^{BH} \cos(2\phi)\\ 
   \alpha_2 (\phi) & = & \frac{x_{A}(1+\epsilon^2)^2}{y}  \left( C_{++}(0) +  
C_{++}(1) cos(\phi) \right)\\
\alpha_3 (\phi) &=& \frac{x^{2}_{A}t(1+\epsilon^2)^2}{y} {\mathcal P}_1(\phi) 
{\mathcal P}_2(\phi) \cdot 2 \frac{2-2y+y^2 + \frac{\epsilon^2}{2}y^2}{1 + 
\epsilon^2}
\end{eqnarray}
\normalsize

Where $S{++}(1)$, $C_{++}(0)$, and $C_{++}(1)$ are the Fourier harmonics in the 
leptonic tensor. Their explicit expressions can be found in Appendix 
\ref{app:Helium_cross_section}. 

Using the $\alpha_{i}$ factors, one can obtain in a model-independent way $\Im
m(\mathcal{H}_{A})$ and $\Re e(\mathcal{H}_{A})$ from fitting the experimental 
$A_{LU}$ as a function of $\phi$ for given values of $Q^2$, $x_B$ and $t$.  



\subsection{Meson Production}

Should include $\pi^0$ and $\phi$.

\section{Past Experiments}
To be written

