
\chapter*{Introduction\markboth{\bf Introduction}{}}
\label{chap:intro}
\addcontentsline{toc}{chapter}{Introduction}
~~~~The electromagnetic probe has been the primary tool for studying the 
internal structure of hadrons in terms of their fundamental constituents, i.e. 
the quarks and the gluons. The leptons are elementary particles characterized 
by their structureless nature, interacting with matter via the well-known 
electromagnetic force and being insensitive to strong interaction. Thus, 
interactions between leptons and hadrons reflect information on the internal 
structure of the target hadrons.

The reaction of interest in this proposal is the so-called Deeply Virtual 
Compton Scattering (DVCS). It is a lepton scattering reaction that gives access 
to structure functions called Generalized Parton Distributions (GPDs). These 
distributions provide a three-dimensional imaging of the partons in a hadron, 
in terms of their longitudinal momentum and transverse spatial distributions.

In 1983, the European Muon Collaboration has discovered that the inclusive DIS 
structure functions of the bound nucleons inside nuclei are different from the 
ones in a free nucleon \cite{EMC_origin}. This effect has been more precisely 
investigated to understand its origins, at CERN, SLAC, DESY(HERMES), Fermilab, 
and JLab. Correlations were established with nuclear properties, such as the 
mass and the nuclear density, but there is still no widely accepted explanation 
for this phenomenon. The nuclear DVCS opens a new avenue to explore the nature 
of medium modifications at the partonic level, generalizing the EMC effect in 
terms of the three-dimensional GPDs instead of the one-dimensional DIS 
structure functions. We can measure two DVCS channels off a given nuclear 
target: the coherent and the incoherent channel. One can measure nuclear GPDs 
from the coherent channel, where the target nucleus remains intact, while from 
the incoherent channel, where the nucleus breaks and the DVCS takes place on a 
bound nucleon, one can access the nucleon GPDs.  

The $^4$He nucleus is of particular interest to study nuclear GPDs as its 
partonic structure is described by only one chirally-even GPD. It is also a 
simple few-body system and has a high density that makes it the ideal target to 
investigate nuclear effects on partons. The Thomas Jefferson National 
Accelerator Facility, known also as Jefferson Lab (JLab), offers unique 
opportunities to perform our reaction of interest as it provides 
longitudinally-polarized high-energy electron beams. The experimental Hall B of 
JLab houses the CLAS12 detector. In this experiment, a 11 GeV 
longitudinally-polarized electron beam was scattered onto a 7 atm $^{4}He$ 
gaseous target. In addition to the CLAS12 detector, A Low Energy Recoil Tracker 
(ALERT), to detect low-energy nuclear recoils, and an Inner Calorimeter (IC), 
to improve the detection of photons at very forward angles, will be used.

This proposal is organized as follows:
\begin{itemize}
\item In chapter 1, the available theoretical tools to study hadronic structure 
   are presented, with an emphasis on the nuclear effects and GPDs.
\item In chapter 2, the characteristics of the CLAS12 spectrometer and the 
working principle of ALERT are discussed.
\item In chapter 3, the identification of the final-state particles and a 
Monte-Carlo simulation are presented.
\item In chapter 4, the selection of the DVCS events, the background 
subtraction, and uncertainty studies on the measured beam-spin asymmetries are 
presented.
\item Finally, a conclusion and beam time request will summarize this proposal.  
\end{itemize} 



