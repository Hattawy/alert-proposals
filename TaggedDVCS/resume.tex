\addcontentsline{toc}{chapter}{Abstract}
{\large\textbf{Run group abstract:}}
\newline

We propose a comprehensive physics program to investigate the quark and
gluon structure of light nuclei, namely deuterium and $^4$He, through coherent and
incoherent exclusive and semi-inclusive Deep Virtual Compton Scattering (DVCS) and
Deep Virtual Meson Production (DVMP). Coherent DVCS off nuclei is a particularly
powerful tool for nuclear tomography through the access of partons position in the
transverse plane. An additional focus of this program is the study of the effect of the
nuclear medium on the structure of nucleons. We propose a next generation nuclear
physics measurements in which low energy recoil nuclei are detected. The tagging of
recoiling nuclei especially in semi-inclusive reactions will be realized for the first time in
these measurements. This powerful technique will provide unique information about the
nature of medium modification including the EMC effect through its dependence on the
nucleon virtuality. Finally, we propose to measure tagged DVCS on light nuclei both for
quasi-free neutron and bound nucleon GPDs. In both cases, we want to study nuclear
effects and their manifestation in GPDs including the effect of Final State Interactions in
the measurements of the bound nucleon beam spin asymmetries.

At the heart of this program is the Low Energy Recoil Tracker (ALERT) in addition
to the CLAS12 detector. The ALERT detector is composed of a stereo drift chamber for
track reconstruction and an array of scintillators for particle identification. Coupling these
two types of fast detectors ensure that ALERT detector can be included in the trigger to
reject background efficiently, while keeping the material budget as low as possible to
detect low energy particles. ALERT will be installed inside the solenoid magnet instead
of the CLAS12 Silicon Vertex Tracker. We will use a gas target straw filled with
deuterium or $^4$He at 3 atm.

The ALERT run group needs 11 GeV longitudinally polarized electron beam.
Although the two main targets are $^4$He and deuterium, we also need to run hydrogen and
$^4$He targets at different beam energies for detector calibration.

\newpage
{\large\textbf{Proposal abstract:}}
\newline

Deeply Virtual Compton Scattering on the proton is set to reveal a 3 dimensional picture
of how quarks and gluons are distributed inside of the nucleon. For light nuclei, the
Fermi motion of the bound nucleon complicates the picture. However, the case of
relatively large Fermi momenta provides an interesting opportunity to study medium
modifications because it corresponds to configuration where two nucleons are separated
by a small distance. In this region, the short-range part of the N-N potential originates
from short-range exchanges whose nature is not well understood.
We propose investigating these configurations with DVCS using $^4$He and
deuterium targets, where the final state includes a recoiling nucleus (less one nucleon)
and the nucleon, which participated in the hard process. The DVCS beam spin
asymmetry on a (quasi-free) neutron will be measured through tagging a recoil proton
from a deuteron target. Similarly, another measurement on the neutron will be performed
by detecting a recoiling triton from a $^4$He target. Taking the ratio of these asymmetries at
fixed DVCS kinematics for different Fermi momenta will provide very important
information on the modification of the nucleon as function of the N-N separation.
We will also measure the impact of final state interactions on incoherent DVCS
measurements to help understand the measurements performed on helium during the
previous CLAS E-08-024 experiment. The measurement of neutron DVCS by tagging
the recoil proton from a deuterium target is complementary to the previously approved
CLAS12 experiment E12-11-003 which will also measure quasi-free neutron DVCS by
detecting the scattered neutron.

\newpage

