
\chapter{Physics Motivations}
\label{chap:physics}

\section{Theory neutron DVCS}

This proposal is to simultaneously explore the incoherent DVCS process:
\begin{itemize}
   \item $^2$H$(e,e^{\prime}\gamma~p)n$,        neutron DVCS ,
   \item $^2$H$(e,e^{\prime}\gamma~^2$H$)$,     deuteron DVCS ,
   \item $^4$He$(e,e^{\prime}\gamma~^2$H$)$NN,  DVCS splitting of helium-4 into deuteron,
   \item $^4$He$(e,e^{\prime}\gamma~^2$H$^2$H), DVCS on deuteron
   \item $^4$He$(e,e^{\prime}\gamma~^2$H$~p)n$, DVCS on deuteron in helium-4,
   \item $^4$He$(e,e^{\prime}\gamma~^3$He$)n$,  neutron DVCS in  helium-4
   \item $^4$He$(e,e^{\prime}\gamma~^3$H$)p$,   proton DVCS in  helium-4
   \item $^4$He$(e,e^{\prime}\gamma~^2$H$p)n$,  proton DVCS in  helium-4
   \item $^4$He$(e,e^{\prime}\gamma~pp)nn$,     proton DVCS in  helium-4
\end{itemize}

\section{Experiment neutron DVCS}

\section{FSI in deuterium}

\section{Theory bound nucleon DVCS}

\section{Experiment bound nucleon DVCS}


\section{Tagged DVCS Reactions}

The ALERT detector combined with CLAS12 provides a unique opportunity  to 
simultaneously explore many novel DVCS reactions on nuclear targets. Tagging a 
wide range of low momentum nuclear recoil  in exclusive knockout reactions will 
yield useful information about final state interactions. Furthermore, the 
over-determined kinematics can be exploited to explore process with unique or 
rare final states.


\subsection{DVCS on the Neutron}

\begin{equation}
e + ^2\text{H} \longrightarrow e^{\prime} + \gamma + p (+ n)
\end{equation}
   $^2$H$(e,e^{\prime}\gamma~p)n$,        neutron DVCS ,

\begin{equation}
   e + ^4\text{He} \longrightarrow e^{\prime} + \gamma + ^3\text{He} (+ n)
\end{equation}
Tagging a recoil helium-3, the reaction $^4$He$(e,e^{\prime}\gamma~^3$He$)n$ 
would be sensitive to the effects of a more tightly bound neutron. In this 
reaction the neutron can be detected forward or selected via missing mass.

\begin{equation}
   e + ^4\text{He} \longrightarrow e^{\prime} + \gamma + ^3\text{He} (+ n)
\end{equation}
Tagging a recoil helium-3, the reaction $^4$He$(e,e^{\prime}\gamma~^3$He$)n$ 
would be sensitive to the effects of a more tightly bound neutron. In this 
reaction the neutron can be detected forward or selected via missing mass.

\subsection{DVCS on the Deuteron}

\subsubsection{Coherent DVCS on the Deuteron}

\begin{equation}
   e + ^2\text{H} \longrightarrow e^{\prime} + ^2\text{H} + \gamma 
\end{equation}

   $^2$H$(e,e^{\prime}\gamma~^2$H$)$,     deuteron DVCS ,

\subsubsection{Incoherent DVCS on the Deuteron}

If only a scattered deuteron is detected from a helium target, the missing mass 
of two nucleons can be used to ensure exclusivity of the reaction 
$^4$He$(e,e^{\prime}\gamma~^2$H$)$NN.

A recoil proton is detected with a forward deuteron
$^4$He$(e,e^{\prime}\gamma~^2$H$~p)n$.

Although the rates are expected to be low, the reaction 
$^4$He$(e,e^{\prime}\gamma~^2$H$^2$H) would be a very interesting subset.

\subsection{DVCS on the proton and recoil breakup}

   $^4$He$(e,e^{\prime}\gamma~^3$He$)n$,  neutron DVCS in  helium-4
   $^4$He$(e,e^{\prime}\gamma~^3$H$)p$,   proton DVCS in  helium-4
   $^4$He$(e,e^{\prime}\gamma~^2$H$p)n$,  proton DVCS in  helium-4
   $^4$He$(e,e^{\prime}\gamma~pp)nn$,     proton DVCS in  helium-4
