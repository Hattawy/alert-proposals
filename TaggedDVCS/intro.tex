
\chapter*{Introduction\markboth{\bf Introduction}{}}
\label{chap:intro}
\addcontentsline{toc}{chapter}{Introduction}

Deeply virtual Compton scattering on a nucleon is an exclusive process which 
when measured over a broad range of kinematics permits a three dimensional 
picture of the nucleon's quark and gluon distributions will emerge. This 
picture emerges from the generalized parton distributions which parameterize 
the cross section. Many measurements have already measured the proton
GPDs, however, without a free neutron target a flavor separation will always 
require nuclear corrections. The EMC effect has demonstrated that medium 
modifications to the nucleon can be significant for heavy nuclei but the degree 
to which modifications exist in light nuclei is unknown.\footnote{Perhaps the 
earliest known medium modification of the nucleon is the free neutron lifetime 
compared the significantly longer lifetime when bound in a nucleus.} 
Furthermore, when considering the Fermi motion of a bound nucleon in deuterium, 
there exist a probability of finding them moving with high relative (and 
opposite) momenta which corresponds to a configuration where they are separated 
by a small distance.  Considering the size of the EMC effect it would not be 
unreasonable to expect these configurations to exhibit signs of modification. 


This proposal consists of three components: nuclear DVCS, incoherent DVCS, 
unique DVCS breakup reactions. First, a comparison of the DVCS beam spin 
asymmetries on the neutron using deuterium and helium-4 targets representing 
the quasi-free and bound nucleon.

Second we will investigate the incoherent DVCS process on 

The final component of this proposal involves exploring novel DVCS reactions 
where the spectator system 


