
\chapter*{Introduction\markboth{\bf Introduction}{}}
\label{chap:intro}
\addcontentsline{toc}{chapter}{Introduction}

Deeply virtual Compton scattering on a nucleon is an exclusive process which 
when measured over a broad range of kinematics can be used to extract 
information about the generalized parton distributions (GPDs) of the nucleon.  
The extracted quark and gluon GPDs offer a three dimensional picture of how 
quarks and gluons are distributed in the nucleon. DVCS measurements on the 
proton\cite{Chekanov:2003ya} have already begun to provide insight into this 
picture of the nucleon, however, without a free neutron target a flavor 
separation will always require an approximation using a deuteron target with 
some nuclear corrections.  The EMC effect demonstrated that medium 
modifications to the nucleon can be significant for heavy nuclei but the degree 
to which modifications exist in light nuclei is unknown.\footnote{Perhaps the 
   earliest known medium modification of the nucleon is the free neutron 
   lifetime compared the significantly longer lifetime when bound in a 
nucleus.} Furthermore, when considering the Fermi motion of a bound nucleon in 
say, deuterium, there exist a probability of finding a nucleon moving with 
large relative momenta which corresponds to a configuration where the two 
nucleons are separated by a small distance.  Although this probability is 
small, hence leading to a small overall contribution to the EMC effect, by 
selecting only these dense configurations through spectator tagging in hard 
processes, sizable modifications of the nucleon can be observed.

of  it would not be unreasonable to expect these configurations to exhibit 
signs of modification. 


This proposal consists of three components: nuclear DVCS, incoherent DVCS, 
unique DVCS breakup reactions. First, a comparison of the DVCS beam spin 
asymmetries on the neutron using deuterium and helium-4 targets representing 
the quasi-free and bound nucleon.

Second we will investigate the incoherent DVCS process on 

The final component of this proposal involves exploring novel DVCS reactions 
where the spectator system 


